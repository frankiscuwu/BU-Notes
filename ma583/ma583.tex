\documentclass{article}

\usepackage{amsmath, amsthm, amssymb, amsfonts}
\usepackage{thmtools}
\usepackage{graphicx}
\usepackage{setspace}
\usepackage[margin=1.5in]{geometry}
\usepackage{float}
\usepackage{hyperref}
\usepackage[utf8]{inputenc}
\usepackage[english]{babel}
\usepackage{framed}
\usepackage[dvipsnames]{xcolor}
\usepackage{tcolorbox}
\usepackage{verbatim}
\usepackage{txfonts}
\usepackage{qtree}

\newcommand{\R}{\mathbb{R}}
\newcommand{\N}{\mathbb{N}}
\newcommand{\Z}{\mathbb{Z}}
\newcommand{\Q}{\mathbb{Q}}
\newcommand{\PP}{\mathbb{P}}
\newcommand{\E}{\mathbb{E}}

\setlength{\parindent}{0pt}

\colorlet{LightGray}{White!90!Periwinkle}
\colorlet{LightOrange}{Orange!15}
\colorlet{LightGreen}{Green!15}

\newcommand{\HRule}[1]{\rule{\linewidth}{#1}}

\declaretheoremstyle[name=Theorem,]{thmsty}
\declaretheorem[style=thmsty,numberwithin=section]{theorem}
\tcolorboxenvironment{theorem}{colback=LightGray}

\declaretheoremstyle[name=Example,]{prosty}
\declaretheorem[style=prosty,numberlike=theorem]{example}
\tcolorboxenvironment{example}{colback=LightOrange}

\declaretheoremstyle[name=Definition,]{prcpsty}
\declaretheorem[style=prcpsty,numberlike=theorem]{definition}
\tcolorboxenvironment{definition}{colback=LightGreen}

\setstretch{1.2}
\geometry{
  textheight=9in,
  textwidth=5.5in,
  top=1in,
  headheight=12pt,
  headsep=25pt,
  footskip=30pt
}

% ------------------------------------------------------------------------------

\begin{document}

% ------------------------------------------------------------------------------
% Cover Page and ToC
\title{ \normalsize \textsc{}
  \\ [2.0cm]
  \HRule{1.5pt} \\
  \LARGE \textbf{\uppercase{CAS MA583 Introduction to Stochastic Processes}
  \HRule{2.0pt} \\ [0.6cm] \LARGE{Spring 2026} \vspace*{10\baselineskip}}
}
\date{}
\author{\textbf{Frank Yang} \\
  Professor Salins \\
MWF 12:20 -- 1:05}

\maketitle
\newpage

\tableofcontents
\newpage

\section{Lecture 1 -- 1/21}
Stocastic := random, process := anything that evolves through time.\\
Examples:
\begin{itemize}
  \item Gambling
  \item Stocks
  \item Biology
  \item Geology
  \item Weather
\end{itemize}

The math used includes calculus (especially infinite sums), linear algebra (solving equations, eigenvalue diffeqs), and probability (know most common distributions including binomial, Poisson, exponential, normal).

\subsection{Chapter 2}
Conditional probability and conditional expectation.
\begin{definition}
  Given two events \(A\) and \(B\) with \(P(B) > 0\), the conditional probability of \(A\) given \(B\) is defined as
  \[\PP(A|B) = \frac{\PP(A \cap B)}{\PP(B)}.\]
\end{definition}
Usually we know the conditional probabilities and use them to solve more difficult questions.
\begin{example}
  Roll a six-sided die, and call the result $X$. Next we flip $X$ coins. Let $Y$ be the total number of coins that land on heads. What is the probability of $\PP(Y=4)$?

  We know some probabilities, like $\PP(X=1) = \frac{1}{6}$, and $\PP(X=2) = \frac{1}{6}$. We also know conditional probabilities, like $\PP(Y=1|X=1) = \frac{1}{2}$, and $\PP(Y=0|X=1) = \frac{1}{2}$. $Y$ is conditionally binomial if $n\leq4$: $\PP(Y=n|X=k) = \binom{k}{n} \left(\frac{1}{2}\right)^k$.

  So to answer the question,
  \begin{align*}
    \PP(Y=4) &= \PP(Y=4 \text{ and } X=4) + \PP(Y=4 \text{ and } X=5) + \PP(Y=4 \text{ and } X=6) \\
    &= \PP(Y=4|X=4)\PP(X=4) + \PP(Y=4|X=5)\PP(X=5) + \PP(Y=4|X=6)\PP(X=6) \\
    &= \binom{4}{4}\left(\frac{1}{2}\right)^4 \cdot \frac{1}{6} + \binom{5}{4}\left(\frac{1}{2}\right)^5 \cdot \frac{1}{6} + \binom{6}{4}\left(\frac{1}{2}\right)^6 \cdot \frac{1}{6}.
  \end{align*}
\end{example}

\begin{definition}
  The above example used what's called the law of total probability.
  \[\PP(Y=n)=\sum_x \PP(Y=n|X=x)\PP(X=x).\]
\end{definition}

\begin{example}
  What is $\E[Y]$ in the above example (or the expected number of heads)? The conditional expectation is clear:
  \[\E[Y|X=4]=\frac{4}{2}=2\implies \E[Y|X=k]=\frac{k}{2}.\]
  Thus the law of total expectation gives us
  \[\E[Y]=\sum_{k=1}^6 \E[Y|X=k]\PP(X=4)=\sum_{k=1}^6 \frac{k}{2} \cdot \frac{1}{6} = \frac{7}{4}.\]
\end{example}

\begin{example}
  Based on the casino game craps: roll two standard six-sided dice over and over. If the sum is 7 then I lose. If the sum is 4, then I win. If the sum is anything else then I roll again. What is the probability that I win?

  Well, in one roll of two dice, the probability of a 4 is $\frac{3}{36}$, the probability of a 7 is $\frac{6}{36}$, and the probability of neither is $\frac{27}{36}$.

  Solving directly without conditionaly, the probability of winning is
  \[\PP = \frac{3}{36} + \frac{27}{36}\frac{3}{36} + \left(\frac{27}{36}\right)^2 \frac{3}{36} + \cdots = \sum_{n=0}^\infty \left(\frac{27}{36}\right)^n \frac{3}{36}.\]

  Using an alternate approach, we could also use the law of total probability. Let $W$ be the event that I win. Then
  \begin{align*}
    \PP(W) &= \PP(W|\text{roll 4})\PP(\text{roll 4}) + \PP(W|\text{roll 7})\PP(\text{roll 7}) + \PP(W|\text{roll other})\PP(\text{roll other}) \\
    &= 1 \cdot \frac{3}{36} + 0 \cdot \frac{6}{36} + \PP(W) \cdot \frac{27}{36}.
  \end{align*}
  Solving for $\PP(W)$ gives
  \[\PP(W) = \frac{3/36}{1 - 27/36} = \frac{1}{9}.\]
\end{example}

\end{document}