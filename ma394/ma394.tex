\documentclass{article}

\usepackage{amsmath, amsthm, amssymb, amsfonts}
\usepackage{thmtools}
\usepackage{graphicx}
\usepackage{setspace}
\usepackage[margin=1.5in]{geometry}
\usepackage{float}
\usepackage{hyperref}
\usepackage[utf8]{inputenc}
\usepackage[english]{babel}
\usepackage{framed}
\usepackage[dvipsnames]{xcolor}
\usepackage{tcolorbox}
\usepackage{verbatim}
\usepackage{txfonts}
\usepackage{qtree}

\newcommand{\R}{\mathbb{R}}
\newcommand{\N}{\mathbb{N}}
\newcommand{\Z}{\mathbb{Z}}
\newcommand{\Q}{\mathbb{Q}}
\newcommand{\PP}{\mathbb{P}}
\newcommand{\E}{\mathbb{E}}
\newcommand{\C}{\mathbb{C}}

\setlength{\parindent}{0pt}

\colorlet{LightGray}{White!90!Periwinkle}
\colorlet{LightOrange}{Orange!15}
\colorlet{LightGreen}{Green!15}

\newcommand{\HRule}[1]{\rule{\linewidth}{#1}}

\declaretheoremstyle[name=Theorem,]{thmsty}
\declaretheorem[style=thmsty,numberwithin=section]{theorem}
\tcolorboxenvironment{theorem}{colback=LightGray}

\declaretheoremstyle[name=Example,]{prosty}
\declaretheorem[style=prosty,numberlike=theorem]{example}
\tcolorboxenvironment{example}{colback=LightOrange}

\declaretheoremstyle[name=Definition,]{prcpsty}
\declaretheorem[style=prcpsty,numberlike=theorem]{definition}
\tcolorboxenvironment{definition}{colback=LightGreen}

\setstretch{1.2}
\geometry{
  textheight=9in,
  textwidth=5.5in,
  top=1in,
  headheight=12pt,
  headsep=25pt,
  footskip=30pt
}

% ------------------------------------------------------------------------------

\begin{document}

% ------------------------------------------------------------------------------
% Cover Page and ToC
\title{ \normalsize \textsc{}
  \\ [2.0cm]
  \HRule{1.5pt} \\
  \LARGE \textbf{\uppercase{CAS MA394 Applied Abstract Algebra}
  \HRule{2.0pt} \\ [0.6cm] \LARGE{Fall 2025} \vspace*{10\baselineskip}}
}
\date{}
\author{\textbf{Frank Yang} \\
  Professor Tom Enkosky \\
TTh 2:00 -- 3:15}

\maketitle
\newpage

\tableofcontents
\newpage

\section{Lecture 1 -- 1/20}
\begin{definition}
  A \textbf{set} is a well-defined collection of objects; that is, it is defined in such a manner that we can determine for any given object whether or not it belonsgs to the set. The objects that belond to a set are called its \textbf{elements} or \textbf{members}. We denote them with capital letters such as $A$ or $X$.

  \begin{itemize}
    \item If $a$ is an element of a set $A$, we write $a\in A$.
    \item The set without any elements its the \textbf{empty set}, denoted $\emptyset$ or $\{\}$.
    \item If $A$ and $B$ are sets and every element of $A$ is in $B$, then $A$ is a \textbf{subset} of $B$, denoted $A\subseteq B$.
    \item If $A\subset B$ and $B\subset A$, then $A=B$.
    \item If $A\subset B$ and $A\neq B$, then $A$ is a \textbf{proper subset} of $B$, denoted $A\subset B$.
  \end{itemize}
  Some standard sets include:
  \begin{itemize}
    \item $\N = \{1, 2, 3, \ldots\}$, the set of natural numbers (positive integers).
    \item $\Z = \{\ldots, -2, -1, 0, 1, 2, \ldots\}$, the set of integers.
    \item $\Q = \left\{\frac{a}{b} \mid a, b \in \Z, b \neq 0\right\}$, the set of rational numbers.
    \item $\R=(-\infty, \infty)$, the set of real numbers.
    \item $\C=\{a+bi \mid a,b\in \R, i^2=-1\}$, the set of complex numbers.
  \end{itemize}
\end{definition}
We can also build new sets from old sets.
\begin{itemize}
  \item The \textbf{union} of two sets $A$ and $B$ is the set of elements that are in $A$ or $B$ (or both), denoted $A\cup B$.
    \[A\cup B=\{x:x\in A\text{ or } x\in B\}.\]
  \item The \textbf{intersection} of two sets $A$ and $B$ is the set of elements that are in both $A$ and $B$, denoted $A\cap B$.
    \[A\cap B=\{x:x\in A\text{ and } x\in B\}.\]
  \item The \textbf{complement} of a set $A$ (with respect to a universal set $U$) is the set of elements in $U$ that are not in $A$, denoted $A',A^c,\bar{A}$.
    \[A'=\{x:x\in U\text{ and } x\notin A\}.\]
  \item The \textbf{difference} of two sets $A$ and $B$ is the set of elements that are in $A$ but not in $B$, denoted $A\setminus B$ or $A-B$.
    \[A\setminus B=A\cap B'=\{x:x\in A\text{ and } x\notin B\}.\]
\end{itemize}
\begin{definition}
  \textbf{De Morgan's Laws} state that for any two sets $A$ and $B$,
  \begin{align*}
    (A\cup B)' &= A' \cap B' \\
    (A\cap B)' &= A' \cup B'.
  \end{align*}
\end{definition}
\begin{definition}
  The \textbf{Cartesian product} of two sets $A$ and $B$ is the set of ordered pairs $(a,b)$ where $a\in A$ and $b\in B$, denoted $A\times B$.
  \[A\times B=\{(a,b):a\in A, b\in B\}.\]
\end{definition}
An example of a Cartesian product is $\R^2=\R\times \R=\{(x,y):x,y\in \R\}$, the set of all points in the plane.
\begin{definition}
  A \textbf{relation} $R$ from a set $A$ to a set $B$ is a subset of the Cartesian product $A\times B$.
\end{definition}
\begin{example}
  Let $A=\{0, 1, 2, 3\}, B=\{0, 1, 2, 3\}$. Then $A\times B=\{(a,b):a,b\in A\}$. If we say that $R$ is the set $\{(0, 1), (0, 2), (0, 3), (1, 2), (1, 3), (2, 3)\}\subset A\times B$, then $R$ is a relation from $A$ to $B$.The relation $R$ can be interpreted as "is less than", since for each $(a,b)\in R$, we have $a<b$.
\end{example}
\begin{definition}
  We define a mapping or function $f\subset A\times B$ from a set $A$ to a set $B$ as a special type of relation where each $a\in A$ has a unique $b\in B$ such that $(a,b)\in f$. We denote this by $f:A\to B$ or $A\xrightarrow{f}B$.
  The set $A$ is called the \textbf{domain} of $f$ and $f(A)=\{f(a):a\in A\}\subset B$ is called the range or \textbf{image} of $f$.
\end{definition}
\section{Lecture 2 -- 1/21}
\begin{definition}
  If $f:A\to B$ is a map and the image of $f$ is $B$, i.e., $f(A)=B$, then $f$ is called \textbf{onto} or \textbf{surjective}. In other words, if there exists an $a\in A$ for every $b\in B$ such that $f(a)=b$, then $f$ is onto.
\end{definition}
\begin{definition}
  A map is $\textbf{one-to-one}$ or \textbf{injective} if $a_1\neq a_2$ implies $f(a_1)\neq f(a_2).$ Equivalently, a function is one-to-one if $f(a_1)=f(a_2)$ implies $a_1=a_2$.
\end{definition}
\begin{definition}
  A map $f:A\to B$ is a \textbf{bijection} if it is both one-to-one and onto.
\end{definition}
\begin{definition}
  Given two functions, we can construct a new function by using the range of the first function as the domain of the second function. Let $f:A\to B$ and $g:B\to C$ be mappings. Define a new map $g\circ f:A\to C$ called the \textbf{composition} of $f$ and $g$ from $A$ to $C$, by $g\circ f(x)=g(f(x))$.
\end{definition}
\begin{itemize}
  \item If $S$ is any set, we use $id_S$ or $id$ to denote the \textbf{identity mapping} from $S$ to itself, or $id(s)=s$ for all $s\in S$.
  \item A map $g:B\to A$ is an \textbf{inverse mapping} of$f:A\to B$ if $g\circ f=id_A$ and $f\circ g=id_B$; in other words, the inverse function of a function simply "undoes" the function.
  \item A map is said to be \textbf{invertible} if it has an inverse. We write $f^{-1}$ to denote the inverse of $f$.
\end{itemize}

\section{Lecture 3 -- 1/22}
\begin{definition}
  An \textbf{equivalence relation} on a set $X$ is a relation $R\subset X \times X$ such that
  \begin{itemize}
    \item $(x,x)\in R$ for all $x\in X$ (reflexive property),
    \item $(x,y)\in R$ implies $(y,x)\in R$ (symmetric property),
    \item $(x,y)\in R$ and $(y,z)\in R$ imply $(x,z)\in R$ (transitive property).
  \end{itemize}
  Given an equivalence relation $R$ on a set $X$, we usually write $x\sim y$ instead of $(x,y)\in\R$. If the equivalence relation already has an associated notation such as $=, \cong, \equiv$, we use that notation instead.
\end{definition}
\begin{example}
  Let $X=\{1, 2, 3, 4\}$. A set $R$ that is reflexive, symmetric, and transitive (and thus an equivalence relation) is $R=\{(1,1), (2,2), (3,3), (4,4), (1,2), (2,1), (1, 3), (3, 1), (2, 3), (3, 2)\}$.
\end{example}
\begin{definition}
  A \textbf{partition} $P$ of a set $X$ is a collection of nonempty sets $X_1, X_2, \ldots, X_n$ such that $X_i\cap X_j=\emptyset$ and $\bigcup_{i} X_i = X$. Let $\sim$ be an equivalence relation on a set $X$ and let $x\in X$. Then $[x]=\{y\in X:x\sim y\}$ is called the \textbf{equivalence class} of $x$.
\end{definition}
\begin{example}
  Let $\Z$ be partitioned into $A=\{\text{even integers}\}$ and $B=\{\text{odd integers}\}$.

  $A\cap B=\emptyset$ and $A\cup B-\Z\implies A,B\text{ partition the integers}$. Define a relation $\sim$ on $\Z$ where $x\sim y\iff x,y$ both even or both odd.

  Every $a\in A$ is equivalent to every other $a'\in A$ and every $b\in B$ is equivalent to every other $b'\in B$.

\end{example}

\section{Lecture 4 -- 1/27}
\subsection{Mathematical Induction}
\begin{definition}
  \textbf{First Principle of Mathematical Induction}: Let $S(n)$ be a statement about integers for $n\in \N$ and suppose $S(n_0)$ is true for some integer $n_0\in\N$. If for all integers $k$ with $k\geq n_0$, $S(k)$ implies that $S(k+1)$ is true, then $S(n)$ is true for all integers $n\geq n_0$.
\end{definition}
\begin{example}
  For all $n\in\N,1+2+3+\cdots+n=\frac{n(n+1)}{2}$.

  The base case is $n=1$:, then $1=\frac{1(1+1)}{2}=1$, so the base case holds.\\
  Inductive hypothesis: Assume $1+2+3+\cdots+n=\frac{n(n+1)}{2}$ for some $n$.\\
  Inductive step: We want to show that the statement is true for $n+1$. Consider $1+2+3+\cdots+n+(n+1)=\frac{n(n+1)}{2}+(n+1)=(n+1)(\frac{n}{2}+1)=(n+1)(\frac{n+2}{2})=\frac{(n+1)((n+1)+1)}{2}$.
\end{example}
\begin{example}
  Prove that
  \[\frac{1}{2}_\frac{1}{6}+\cdots+\frac{1}{n(n+1)}=\frac{n}{n+1}\]
  for $n\in\N$.

  Base: $n=1$, then $\frac{1}{1\cdot 2}=\frac{1}{2}$, so the base case holds.\\
  Inductive hypothesis: Assume $\frac{1}{2}+\frac{1}{6}+\cdots+\frac{1}{n(n+1)}=\frac{n}{n+1}$ for some $n$.\\
  Inductive step: We want to show that the statement is true for $n+1$. Consider
  \begin{align*}
    \frac{1}{2}+\frac{1}{6}+\cdots+\frac{1}{n(n+1)}+\frac{1}{(n+1)(n+2)}&=\frac{n}{n+1}+\frac{1}{(n+1)(n+2)}\\&=\frac{n(n+2)+1}{(n+1)(n+2)}\\&=\frac{(n+1)^2}{(n+1)(n+2)}\\&=\frac{n+1}{n+2}.
  \end{align*}
  Therefore the statement is true for all $n$.
\end{example}

\subsection{The Division Algorithm}
\begin{definition}
  Let $a$ and $b$ be integers with $b>0$. Then there exist unique integers $q$ and $r$ such that $a=bq+r$ where $0\leq r<b$.

  \begin{itemize}
    \item Let $a$ and $b$ be integers. If $b=ak$ for some integer $k$, we write $a|b$.
    \item An integer $d$ is called a \textbf{common divisor} of $a$ and $b$ if $d|a$ and $d|b$.
    \item The \textbf{greatest common divisor} of $a$ and $b$ is a positive integer $d$ such that $d$ is a common divisor of $a$ and $b$, and if $d'$ is any other common divisor of $a$ and $b$, then $d'|d$. We write $d=gcd(a,b)$.
    \item We say that two integers $a$ and $b$ are relatively prime if $gcd(a,b)=1$.
  \end{itemize}

\end{definition}
Proof of the $gcd$ theorem uses the division algorithm:
\begin{align*}
  a &= q_1b+r_1\;, 0\leq r_1 < b\\
  b&=q_2r_1+r_2\;, 0\leq r_2<r_1, r_1\neq0\\
  r_1&=q_3r_2+r_3\;, 0\leq r_3<r_2, r_2\neq0\\
  &\vdots\\
  r_{n-2}&=q_nr_{n-1}+r_n\;, 0\leq r_n<r_{n-1}, r_{n-1}\neq0\\
  r_{n-1}&=q_{n+1}r_n+0.
\end{align*}
The first non-zero remainder is $r_n|gcd(a,b)$, or a common divisor. To show that it's the greatest common divisor, suppose $d=as'+bt'$ for some $s',t'\in\Z$ where $d=gcd(a,b)$. Then if $a=q_1b+r_1$ and $b=q_2r_1+0$, if $d>r$, but $d$ is a common divisor, then $a-q_1b=r_1$. Thus $d|a,d_b\implies d|r_1$, therefor $d$ can't be bigger than $r_1.$
\begin{example}
  Let $a=23, b=3$. Since $23=1\cdot3+20$, we have $q=1$ and $r=20$ which is not valid since $r$ is not less than $b$. We also have $23=8\cdot3-1$, so $q=8$ and $r=-1$ is not valid since $r$ is not nonnegative.

  The only valid one is $23=3\cdot7+2$, so $q=7$ and $r=2$. There is no other way to write $23=3\cdot q+r$ where $0\leq r<3$.
\end{example}

\begin{example}
  Use the Division Algorithm to find $gcd(2520, 378)$.
  \begin{align*}
    2520&=378(6)+252\\
    378&=252(1)+126\\
    252&=126(2)+0
  \end{align*}
  The last non-zero remainder is the gcd, so $gcd(2520, 378)=126$.
\end{example}
\section{Lecture 5 -- 1/28}
\begin{theorem}
  Let $p$ be an integer such that $p>1$. We say that $p$ is a \textbf{prime number}, or simply $p$ is \textbf{prime}, if the only positive divisors of $p$ are 1 and $p$ itself. An integer greater than 1 that is not prime is called a \textbf{composite number}.
\end{theorem}
Lemma. Euclid. Let $a$ and $b$ be integers and $p$ be a prime number. If $p|ab$, then either $p|a$ or $p|b$. Suppose $p|ab$ and suppose $p\nmid a$. Then $gcd(a,p)=1$. Thus there exist integers $s,t$ such that $as+pt=1$. Multiplying both sides by $b$ gives $abs+ptb=b$. Since $p|ab$ and $p|ptb$, we have $p|b$.
\begin{theorem}
  There exist an infinite number of primes.

  Assume $p_1, p_2, \cdots, p_n$ are all the primes. Let $N=p_1p_2\cdots p_n+1$. Then $N$ is either prime or composite. If $N$ is prime, then there exists a prime not in our list. If $N$ is composite, then it has a prime divisor $p$. Since $p$ divides $N$ and $p$ divides $p_1p_2\cdots p_n$, $p$ must also divide their difference, which is 1. This is a contradiction since no prime divides 1. Thus there exists a prime not in our list.
\end{theorem}
\subsection{3.1 Integer Equivalence Classes and Symmetries}
$\Z_n=\{[0], [1], [2],\cdots,[n-1]\}$ where arithmetic is mod $n$ and the equivalence classes are $a\equiv b(mod\ n)$ if $n|a-b$.
\begin{example}
  $\Z_6:$\\
  $[0]=\{0, \pm6, \pm12, \pm18\cdots\}=[6]=[-12]$ all have a remainder of 0 when divided by 6.\\
  $[1]=\{1, \pm5, \pm11, \pm17\cdots\}=[7]=[-11]$ all have a remainder of 1 when divided by 6.\\
  $\vdots$
\end{example}
\section{Lecture 6 -- 1/29}
\begin{definition}
  Let $\Z_n$ be the set of equivalence classes of the integers $mod\ n$ and $a,b,c\in\Z_n$.

  \begin{enumerate}
    \item Addition and multiplication are commutative. $ab\equiv ba$ and $a+b\equiv b+a$.
    \item Addition and multiplication are associate. $(ab)c\equiv a(bc)$ and $(a+b)+c\equiv a+(b+c)$.
    \item There are both additive and multiplicative identities. $0\in \Z_n$ has the property $a+0\equiv a$ and $1\in \Z_n$ has the property $a\cdot1\equiv a$.
    \item Multiplication distributes over addition. $a(b+c)\equiv ab+ac$.
    \item For every integer $a$ there is an additive inverse $-a$. $a+(n-a)\equiv (n-a)+a\equiv n\equiv0$.
    \item Let $a$ be a nonzero integer. Then $gcd(a,n)=1$ if and only if there exists a multiplicative $b$ inverse for $a(mod\ n)$.
  \end{enumerate}
\end{definition}
\begin{definition}
  A \textbf{symmetry} of a geometric figure is a rearrangement of the figure preserving the arrangement of its sides and vertices as well as its distances and angles. A map from the plane to itself preserving the symmetry of an object is called a \textbf{rigid motion}.
\end{definition}
\subsection{3.2 Definitions and Examples}
\begin{definition}
  A \textbf{binary operation} or \textbf{law of composition} on a set $G$ is a function $G\times G\to G$ that assigns to each pair $(a,b)\in G\times G$ a unique element $a \circ b$, or $ab$ in $G$, called the composition of $a$ and $b$.
\end{definition}
\begin{definition}
  A \textbf{group} is a set $G$ together with a law of composition $\circ$ that satisfies the following axioms:
  \begin{itemize}
    \item The law of composition is \textbf{associative}; that is, for all $a,b,c\in G$, we have $(a\circ b)\circ c=a\circ(b\circ c)$.
    \item There exists an \textbf{identity element} $e$ in $G$ such that for all $a\in G$, we have $e\circ a=a\circ e=a$.
    \item For each element $a\in G$, there exists an \textbf{inverse element} $a^{-1}\in G$ such that $a\circ a^{-1}=a^{-1}\circ a=e$.
  \end{itemize}
\end{definition}
A group with the property that $a\circ b=b\circ a$ for all $a,b\in G$ is called an \textbf{abelian group} or \textbf{commutative group}. Groups not satisfying this property are called \textbf{nonabelian} or \textbf{non-commutative}.

\section{Lecture 7 -- 2/3}

The identity element in a group $G$ is unique; that is, there exists only one element $e\in G$ such that $eg=ge=g$ for all $g\in G$.

If $g$ is any element in a group $G$, then the inverse of $g$, denoted by $g^{-1}$, is unique.

Let $G$ be a group. If $a,b\in G$, then $(ab)^{-1}=b^{-1}a^{-1}$.

Let $G$ be a group. For any $a\in G$, $(a^{-1})^{-1}=a$.

Let $G$ be a group and $a$ and $b$ be any two elements in $G$. Then the equations $ax=b$ and $xa=b$ have unique solutions in $G$.

If $G$ is a group and $a,b,c\in G$, then $ba=ca$ implies $b=c$ and $ab=ac$ implies $b=c$.

\begin{theorem}
  In a group, the usual law of exponents hold; that is, for all $g,h\in G$,
  \begin{enumerate}
    \item $g^mg^n=g^{m+n}$ for all $m,n\in\Z$.
    \item $(g^m)^n=g^{mn}$ for all $m,n\in\Z$.
    \item $(gh)^n=(h^{-1}g^{-1})^{-n}$ for all $n\in\Z$. Furthermore, if $G$ is abelian, then $(gh)^n=g^nh^n$.
  \end{enumerate}
\end{theorem}
\end{document}
